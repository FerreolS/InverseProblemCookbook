%      Freely inspired from https://www.charles-deledalle.fr/pages/files/cookbook_datascience.pdf
%      Copyright (C) Ferréol Soulez, 2020 
%      ferreol.soulez@univ-lyon1.fr
%
%     This program is free software: you can redistribute it and/or modify
%     it under the terms of the GNU General Public License as published by
%     the Free Software Foundation, either version 3 of the License, or
%     (at your option) any later version.
%
%     This program is distributed in the hope that it will be useful,
%     but WITHOUT ANY WARRANTY; without even the implied warranty of
%     MERCHANTABILITY or FITNESS FOR A PARTICULAR PURPOSE.  See the
%     GNU General Public License for more details.
%
%     You should have received a copy of the GNU General Public License
%     along with this program.  If not, see <http://www.gnu.org/licenses/>.
%
%% Ferréol Soulez, 2020

\documentclass[a4paper]{cookbook}


\usepackage{amsmath,amsfonts,amssymb,textcomp,mathtools}
\usepackage{mathrsfs}
\usepackage{xspace} 


\makeatletter \newcommand{\MathFuncName}[1]{{\operator@font #1}}
\newcommand{\MathFunc}[1]{\mathop{\operator@font #1}\nolimits}
\newcommand{\MathFuncWithLimits}[1]{\mathop{\operator@font #1}\limits}
\makeatother

\newcommand{\asterisk}{*}
\newcommand{\mathd}{\mathrm{d}} % roman 'd' for integrand/differential
\newcommand{\mathe}{\mathrm{e}} % roman 'e' for exp(1)
\newcommand{\mathi}{\mathrm{i}} % roman 'i' for sqrt(-1)

\newcommand{\BF}[1]{\mathbf{#1}} % math bold
\newcommand{\RM}[1]{\mathrm{#1}} % math roman
\newcommand{\SF}[1]{\mathsf{#1}} % math sans-serif
\newcommand{\OP}[1]{{\MathFuncName{#1}}} % upright (*) for function names
\newcommand{\Tag}[1]{\mathsf{#1}} % math font for labels and tags
\newcommand{\BS}[1]{{\boldsymbol{#1}}} % bold symbol (*)
\newcommand{\V}[1]{\boldsymbol{#1}} % vector
\newcommand{\M}[1]{\mathbf{#1}} % matrix
\newcommand{\Inner}[2]{\left \langle #1,#2 \right\rangle} % Inner product
% \newcommand{\TransposeLetter}{\mathrm{T}}
% \newcommand{\TransposeLetter}{\mathsf{t}}
% \newcommand{\TransposeLetter}{{\intercal}}
% \newcommand{\TransposeLetter}{\top}
\newcommand{\AdjointLetter}{\dagger}
\newcommand{\T}{^{\AdjointLetter}} % suffix for transpose
\newcommand{\TransposeLetter}{\top}
\newcommand{\Tt}{^{\TransposeLetter}} % suffix for transpose
\newcommand{\mT}{^{-\TransposeLetter}} % suffix for inverse transpose
\newcommand{\I}{\mathi} % roman 'i' for sqrt(-1)
\newcommand{\E}{\mathe} % roman 'e' for exp(1)
\newcommand{\D}[1]{{\mathd #1}} % integrand/differential
\newcommand{\ScaleAs}{\mathcal{O}} % O(n)
\renewcommand{\Re}{\MathFunc{Re}} % real part
\newcommand{\sign}{\MathFunc{sign}} % sign
\newcommand{\Arg}{\MathFunc{Arg}} % argument
%\newcommand{\mod}{\MathFunc{mod}} % modulo
\renewcommand{\Im}{\MathFunc{Im}} % imaginary part
\newcommand{\arc}{\MathFunc{arc}} % lenght of arc
\newcommand{\rect}{\MathFunc{rect}} % real part
\newcommand{\rnd}{\MathFunc{rnd}} % random
\newcommand{\sgn}{\MathFunc{sgn}} % sign function
\newcommand{\sinc}{\MathFunc{sinc}} % cardinal sine
\newcommand{\asin}{\MathFunc{asin}} % inverse sine
\newcommand{\acos}{\MathFunc{acos}} % inverse cosine
\newcommand{\Var}{\MathFunc{Var}} % variance
\newcommand{\Cov}{\MathFunc{Cov}} % covariance
\DeclarePairedDelimiterXPP{\Expect}[1]{\mathrm{E}}(){}{#1}
\newcommand*{\delimsize}{}
\newcommand*{\given}{\,\vert\,} % compact version
\newcommand*{\Given}{\:\delimsize\vert\:} % autoscale to surrounding version

\newcommand{\Card}{\MathFunc{Card}} % cardinal, number of elements
\newcommand{\diag}{\MathFunc{diag}} % diagonal
\newcommand{\eig}{\MathFunc{eig}} % eigenvalues
\newcommand{\tr}{\MathFunc{tr}} % trace
\newcommand{\rank}{\MathFunc{rank}} % rank
\newcommand{\pdf}{\MathFunc{pdf}} % probability density function
\newcommand{\para}{{/\!/}} % parallel
\newcommand{\prox}{\operatorname{prox}} % proximity operator
\newcommand{\conj}[1]{\operatorname{conj}\Paren{#1}} % proximity operator
\newcommand{\proxy}[1]{\widetilde{#1}} % approximation
% (*) enclosed in braces so that x_\BS{u} or x_\RM{name} works
\newcommand{\comp}{\mathop{\circ}}



% Partial derivative:
\newcommand{\PDer}[2]{\frac{\partial #1}{\partial #2}}
\newcommand{\At}[2]{\left.#1\right\vert_{#2}}

\newcommand{\Paren}[1]{\left(#1\right)}
\newcommand{\bigParen}[1]{\bigl(#1\bigr)}
\newcommand{\BigParen}[1]{\Bigl(#1\Bigr)}

\newcommand{\Brace}[1]{\left\{#1\right\}}
\newcommand{\bigBrace}[1]{\bigl\{#1\bigr\}}
\newcommand{\BigBrace}[1]{\Bigl\{#1\Bigr\}}

\newcommand{\SqBrack}[1]{\left[#1\right]}
\newcommand{\bigSqBrack}[1]{\bigl[#1\bigr]}
\newcommand{\BigSqBrack}[1]{\Bigl[#1\Bigr]}

% Norm, e.g. ||x|| (fixed and scalable):
\newcommand{\norm}[1]{\Vert #1\Vert} 
\newcommand{\Norm}[1]{\left\Vert #1\right\Vert}
\newcommand{\bigNorm}[1]{\bigl\Vert #1\bigr\Vert}
\newcommand{\BigNorm}[1]{\Bigl\Vert #1\Bigr\Vert}

% Absolute value, e.g. |x| (fixed and scalable):
\newcommand{\abs}[1]{\vert #1\vert} 
\newcommand{\Abs}[1]{\left\vert #1\right\vert} \newcommand{\bigAbs}[1]{\bigl\vert #1\bigr\vert}
\newcommand{\BigAbs}[1]{\Bigl\vert #1\Bigr\vert}

% Average value, e.g. <x> (fixed and scalable):
\newcommand{\avg}[1]{\langle #1\rangle}
\newcommand{\Avg}[1]{\left\langle #1\right\rangle}
\newcommand{\bigAvg}[1]{\bigl\langle #1\bigr\rangle}
\newcommand{\BigAvg}[1]{\Bigl\langle #1\Bigr\rangle}

% Complex quantity:
\newcommand{\complex}[1]{\underline{#1}}

% For optimization problems, "arg min" and "arg max":
\newcommand{\argmin}{\MathFuncWithLimits{arg\,min}}
\newcommand{\argmax}{\MathFuncWithLimits{arg\,max}}
\newcommand{\minimize}{\MathFuncWithLimits{minimize}}
\newcommand{\find}{\MathFuncWithLimits{find}}

% Marker for end of proof/algorithm:
\newcommand{\EndProof}{\ensuremath{\blacksquare}}
% \newcommand{\EndProof}{\ensuremath{Box{}}} Quadratic term:
\newcommand{\QuadTerm}[2]{#2\T\!\!\cdot #1\cdot #2}

% Fourier transforms:
\newcommand{\FT}[1]{\widehat{#1}} % Fourier transform
\newcommand{\TwoIPi}{2\,\I\,\pi} % 2*i*pi
\newcommand{\F}[1]{\mathcal{F}\Paren{#1}}
\newcommand{\IF}[1]{\mathcal{F}^{-1}\Paren{#1}}
\newcommand{\FTarrow}{\xrightarrow{\mathrm{FT}}}
\newcommand{\DFTarrow}{\xrightarrow{\mathrm{DFT}}}

% Sets:
\newcommand{\Set}[1]{\mathbb{#1}} \newcommand{\Reals}{\mathbb{R}}
\newcommand{\Integers}{\mathbb{Z}} \newcommand{\Natural}{\mathbb{N}}
\newcommand{\Complexes}{\mathbb{C}}
\newcommand{\Rationals}{\mathbb{Q}}

% By definition: \newcommand\bydef{\equiv}
% \newcommand\bydef{\triangleq}
\newcommand{\bydef}{\stackrel{\text{\tiny def}}{=}}

\newcommand\BesselJ[1]{\mathop{\mathrm{J}}\nolimits_{#1}}

\usepackage{lipsum}                                % Dummy text
\usepackage[figwidth = 0.98\linewidth]{todonotes}  % Dummy image (and more!)
\usepackage[absolute, overlay]{textpos}            % Figure placement
\setlength{\TPHorizModule}{\paperwidth}
\setlength{\TPVertModule}{\paperheight}

\usepackage{tabls}

\title{Selected recipes for inverse problems}
\author
{%
    Ferréol Soulez
}
%\institute{Centre de Recherche Astrophysique de Lyon}


%% Remove footline:
%\setbeamertemplate{footline}{}


\begin{document}
\begin{frame}

\vspace{-1.5em}    
\begin{columns}[onlytextwidth]


\begin{column}{0.5\textwidth - .5cm}
    \begin{exampleblock}{Notations}
        \begin{tabular*}{\columnwidth}{@{\extracolsep{\fill}}l r }
            $\alpha$ &  (lower case) scalar \\
            $\alpha^*$ & complex conjugate of $\alpha$\\
            $\V{x} \in \Complexes^N$ &  (boldface lower case) complex vector\\
            $N$ &   (upper case) cardinal (number of elements)\\
            $\M{H} \in \Complexes^{N\times M}$ & (boldface upper case) linear operator (matrix) \\%  $\Complexes^N \to \Complexes^M$\\
            $f: \Complexes^N \to \Complexes^M $ & (lower case) function  \\
            $ L^{2}(\Reals^N)$ & space of squared-integrable function \\
            $\mathcal{H}:  L^{2}(\Reals^N) \to L^{2}(\Reals^M)$ & (calligraphic upper case)  operator \\ & acting on functions
    \end{tabular*}
    \end{exampleblock}
    
    \begin{block}{Basic properties}
    {\tablinesep=1.2ex
        \begin{tabular*}{\columnwidth}{@{\extracolsep{\fill}}l r }
        $\left[\M{H}\,\V{x}\right]_m = \sum_n H_{m,n}\,x_n$ &  vector matrix product\\
        $\left[\M{H}\,\M{G}\right]_{m,k} = \sum_{n} H_{m,n}\,G_{n,k}$ &  matrix product\\
        $\M{H}\left(\alpha\,\V{x}+\beta\,\V{y}\right) = \alpha\,\M{H}\,\V{x}+\beta\,\M{H}\,\V{y}$ & linearity\\ 
        $\left({\V{x}}\times {\V{y}}\right)_n = x_n\,y_n$ & element wise product \\
        $\M{I}  =    
\begin{bmatrix}
1\ & 0\ & 0\ &\dots &0\\
0&1&0&\dots &0\\
0&0&1&\dots &0\\
\vdots &\vdots &\vdots &\ddots &\vdots \\
0&0&0&\dots &1\end{bmatrix}$ & identity matrix\\
    \end{tabular*}}
    {\tablinesep=1.2ex
        \begin{tabular*}{\columnwidth}{@{\extracolsep{\fill}}l r }
        $\M{A} = \diag(\V{a}) =    
\begin{bmatrix}
a_0&0&0&\dots &0\\
0&a_1&0&\dots &0\\
0&0&a_2&\dots &0\\
\vdots &\vdots &\vdots &\ddots &\vdots \\
0&0&0&\dots &a_{N-1}\end{bmatrix}$ & diagonal matrix % $\V{a}= \begin{bmatrix} a_0\\ a_1\\ a_2\\ \vdots \\ a_{N-1} \end{bmatrix}$ 
    \end{tabular*}}
    \end{block}
    
    
    \begin{block}{Adjoint, scalar product}
    {\tablinesep=1.2ex
        \begin{tabular*}{\columnwidth}{@{\extracolsep{\fill}}l r }
        $\left(H\T\right)_{i,j} = H^*_{j,i} $ &  adjoint (transpose conjugate)\\
        $\Inner{\V{x}}{\V{y}} = \sum_{n=0}^{N-1} x_n^*\,y_n = \V{x}\T \V{y}$ &  dot (or scalar or inner) product\\
         $\Inner{\M{H}\,\V{x}}{\V{y}} = \Inner{\V{x}}{\M{H}\T\,\V{y}}$ & adjoint formal definition\\
         $\left(\alpha\,\M{H} + \beta\,\M{G}\right)\T = \alpha^*\,\M{H}\T + \beta^*\,\M{G}\T$&\\
         $\left(\M{H}\,\M{G}\right)\T = \M{G}\T\,\M{H}\T$&
    \end{tabular*}}
    \end{block}
    
    \begin{block}{Norms}
    {\tablinesep=1.5ex
        \begin{tabular*}{\columnwidth}{@{\extracolsep{\fill}}l r }
        $  \Norm{\V{x}}_2^2  =  \Inner{\V{x}}{\V{x}} = \V{x}\T\V{x} = \sum_{n=0}^{N-1} \Abs{x_n}^2 = \sum_{n=0}^{N-1} x_n^*\,x_n$ & $\ell^2$ norm
    \end{tabular*}}
    {\tablinesep=1.2ex
        \begin{tabular*}{\columnwidth}{@{\extracolsep{\fill}}l r }
        $ \Abs{ \Inner{\V{x}}{\V{y}}} \leq   \Norm{\V{x}}_2 \, \Norm{\V{y}}_2 $ & (Cauchy-Schwartz)   \\
        $\Norm{\V{x} + \V{y} }^2_2 = \Norm{\V{x}}^2_2 +  \Norm{\V{y}}^2_2 + 2\, \Inner{\V{x}}{\V{y}} $ & \\
        $  \Norm{\V{x}}_p^p  =\sum_{n=0}^{N-1} \Abs{x_n}^p\ , \,p\geq 1 $ & $\ell^p$ norm\\
        $ \Norm{\V{x} + \V{y} }_p \leq \Norm{\V{x}}_p +  \Norm{\V{y}}_p$  & triangular inequality\\
        $\Norm{\M{H}}^2_F = \sum_{n,m} \Abs{H_{i,j}}^2 = \tr(\M{H}\T\M{H}) $& Frobenius norm  
        \end{tabular*}}
    \end{block}
 
    \begin{block}{Inverse}
    $\M{H}\in \Complexes^{N\times N}$ is called invertible (also nonsingular or nondegenerate) if there exists a matrix $\M{G}$ such that:
    \begin{align*}
        \M{H} \M{G} = \M{G} \M{H} =  \M{I}
    \end{align*}
    $\M{G} =  \M{H}^{-1}$ is unique and is the inverse of $\M{H}$\\
    {\tablinesep=1.1ex
        \begin{tabular*}{\columnwidth}{@{\extracolsep{\fill}}l r }
         $\left(\M{H}\,\M{M}\right)^{-1} = \M{M}^{-1}\,\M{H}^{-1}$& for any $\M{H}\,,\M{M}$ invertible\\
         $\left(\M{H}\T\right)^{-1} = \left(\M{H}^{-1}\right)\T$&  
        \end{tabular*}}
    \end{block}
 


\end{column}


\begin{column}{0.5\textwidth - .5cm}

    
    \begin{block}{Eigenvalue \& eigenvectors of  $\M{H} \in \Complexes^{N\times N}$}
    $\lambda_i\in \Complexes$ and $\V{v}_i\in \Complexes^{N}$ are the $i^\text{th}$ eigenvalue and eigenvector  respectively. They satisfy:
    \begin{align*}
        \M{H}\,\V{v}_i = \lambda_i\,\V{v}_i
    \end{align*}
    It leads to the eigendecomposition:
    \begin{align*}
         \M{H} = \M{Q}\,\diag(\V{\lambda})\,\M{Q}^{-1}\,,
    \end{align*}
    where the columns of $\M{V}$ are the $N$ eigenvectors of $\M{H}$.
    \begin{itemize}
        \item $\V{\lambda} = \eig(\M{H})$ is the spectrum of $\M{H}$,
        \item $C = \frac{\max\Abs{\V{\lambda}}}{\min\Abs{\V{\lambda}}}$ is the condition number,
        \item $\rank(\M{H})$ is the  number of non-zero element of  $\V{\lambda}$,
        \item if $\lambda_i\neq 0,\,\forall i$,  $\M{H} $ is invertible  and  $\M{H}^{-1} = \M{Q}\,\diag(\V{\lambda})^{-1}\,\M{Q}^{-1}\,$,
    \end{itemize}
    \end{block}
    

    \begin{block}{Peculiar matrices}
    \begin{itemize}
        \item {\bf Unitary:} $\M{Q}\T\, \M{Q} = \M{Q}\, \M{Q}\T = \M{I}$
        \item {\bf Hermitian:} $\M{H}\T= \M{H}$ and  $\M{H} = \M{Q}\,\diag(\V{\lambda})\,\M{Q}\T$ with $\V{\lambda}\in\Reals^N$ and $\M{Q}$ unitary
        \item {\bf Positive semi-definite:} $ \lambda_i\geq 0\,,\ \forall i \iff \V{x}\T\M{H}\V{x} \geq 0, \, \forall \V{x}$ 
        \item {\bf Toeplitz:} $H_{i,j} = h_{i-j}$,
        \item {\bf Circulant:} Toeplitz with  $H_{i,j}= h_{(i-j) \mod N}$
    \end{itemize}
    \end{block}
    
    \begin{block}{Trace and determinant} 
    \begin{itemize}
        \item {\bf  trace:}   $\ \tr(\M{H}) = \sum_n H_{n,n} = \sum_n \lambda_n $
    \end{itemize}
         {\tablinesep=1.1ex
        \begin{tabular*}{\columnwidth}{@{\extracolsep{\fill}}l r }
         $\tr(\M{H}\T) = \tr(\M{H})^*$&  $\tr(\M{H}\,\M{G}) =\tr(\M{G}\,\M{H})$\\
         $\tr(\alpha\,\M{H}+\beta\,\M{G}) =\alpha\,\tr(\M{H})+\alpha\,\tr(\M{H})$& \\
          \end{tabular*}}
    \begin{itemize}
        \item {\bf  determinant:}  $\ \det(\M{H}) =  \prod_n \lambda_n  $
    \end{itemize}
    {\tablinesep=1.1ex
        \begin{tabular*}{\columnwidth}{@{\extracolsep{\fill}}l r }
         $\det(\M{H}) \neq 0 \iff \M{H} $ is invertible &   $\det(\M{H}\T) = \det(\M{H})^*$\\
         $\det(\M{H}\,\M{G}) =\det(\M{H})\,\det(\M{G})$& $\det(\M{H}^{-1})= 1/ \det(\M{H})$
        \end{tabular*}}
    \end{block}
    
    
    \begin{block}{Singular value decomposition}
    For all matrices $\M{H}\in \Complexes^{N\times M}$ there exists two unitary matrices $\M{U} \in \Complexes^{N\times N}$ and  $\M{V} \in \Complexes^{M\times M}$ and a real non-negative diagonal matrix $\M{\Sigma} \in  \Complexes^{N\times M}$ (singular values: $\sigma_i = \Sigma_{i,i}\,,\ \forall i\leq \min(M,N)$)  subject to:
    \begin{align*}
        \M{H} = \M{U}\,\M{\Sigma}\,\M{V}\T
    \end{align*}
    \vspace{-2em}\begin{itemize}
        \item  columns of $\M{U}$  are eigenvectors of $\M{H}\,\M{H}\T$,  
        \item columns of  $\M{V}$ are eigenvectors of $\M{H}\T\,\M{H}$,  
        \item $\M{\Sigma} = \sqrt{\diag\left(\eig\left(\M{H}\,\M{H}\T\right)\right)} $,
        \item $\Norm{\M{H}}^2_F = \sum_{i} \sigma_i^2 $
        \item $\rank(\M{H})$: number of non zero singular values,
    %    \item if $\M{H}$ is Hermitian $\M{U} = \M{V} = \M{Q}$
    \end{itemize}
    \end{block}

    \begin{block}{Inversion lemmas \& Woodbury identity}
    $ \M{B}^{-1}\M{V} \left(\M{A} - \M{U}\M{B}^{-1}\M{V}\right)^{-1} = \left( \M{B} - \M{V}\M{A}^{-1}\M{U}\right)^{-1}\M{V}\M{A}^{-1}$\\
    $  \left(\M{A} - \M{U}\M{B}^{-1}\M{V}\right)^{-1}\M{U}\M{B}^{-1} = \M{A}^{-1}\M{U}\left( \M{B} - \M{V}\M{A}^{-1}\M{U}\right)^{-1}$\\
    $\left(\M{A} + \M{U}\,\M{B}\,\M{V} \right)^{-1} = \M{A}^{-1} - \M{A}^{-1}\,\M{U}\left( \M{B}^{-1} + \M{V}\,\M{A}^{-1}\,\M{U}\right)^{-1}\,\M{V}\,\M{A}^{-1}$
    \end{block}
    
    \begin{block}{Moore-Penrose pseudo inverse}
    The pseudo inverse of $\M{H} = \M{U}\,\M{\Sigma}\,\M{V}\T$ writes  $\M{H}^+ = \M{V}\,\M{\Sigma}^+\,\M{U}\T$ with $\Sigma_{i,i}^+ = \left\{
                            \begin{array}{ll}
                              \Sigma^{-1}_{i,i} &   \text{if }\sigma_i\neq0 \,, \\
                              0 & \text{otherwise} \,.
                            \end{array}
                                  \right.$ 
                                  
   \begin{itemize}
       \item        $\M{H}\,\M{H}^+\,\M{H} = \M{H} $ and   $ \M{H}^+\,\M{H}\,\M{H}^+ = \M{H}^+ $
         \item   $\M{H}$ is square and   $\ \rank(\M{H}) = N \Rightarrow \M{H}^+ = \M{H}^{-1}$
        \item     $\M{H}$ is broad: $\ \rank(\M{H}) \leq N$ and $ \M{H}^+ = \M{H}\T \left(\M{H}\,\M{H}\T\right)^{-1}$
           \item $\M{H}$ is  tall: $\ \rank(\M{H}) \leq M $ and $\M{H}^+ = \left(\M{H}\T\,\M{H}\right)^{-1}\,\M{H}\T $ 
   \end{itemize}
    \end{block}

\end{column}


\end{columns}

\end{frame}

\begin{frame}
\vspace{-1.5em}
\begin{columns}[onlytextwidth]
\begin{column}{0.5\textwidth - .5cm}
    \begin{block}{Derivatives} 
      {\tablinesep=1.1ex
        \begin{tabular*}{\columnwidth}{@{\extracolsep{\fill}}l r }
         $J_{i,j}=\left[\PDer{\V{x}}{\V{y}}\right]_{i,j} = \PDer{x_i}{y_j}  $ & Jacobian matrix\\
        $\left[\nabla f \right]_i= \PDer{f}{x_i}$ &    Gradient vector\\
        $\left[\nabla^2 f\right]_{i,j} = 
\frac{\partial^2 f}{\partial x_i\,\partial x_j }$ &    Hessian matrix
        \end{tabular*}}
    \end{block}
    \begin{block}{Derivation rules} 
      {\tablinesep=1.1ex
        \begin{tabular*}{\columnwidth}{@{\extracolsep{\fill}}l r }
     %   $\partial \M{A} = 0$ \\
        $\partial\left(\alpha\, \M{A}+ \beta \,\M{B}\right) = \alpha\,\partial\M{A} +\beta\,\partial\M{B}$ & linearity \\
        $\partial\left( \M{A}\,\M{B}\right) =\left( \partial\M{A}\right) \, \M{B} +  \M{A}\, \left(\partial\M{B}\right)$  &  \\
        $\partial\left( \M{A}^{-1}\right) = -\M{A}^{-1}\left( \partial\M{A}\right)\,\M{A}^{-1}$ &  \\
        $\partial\left( \M{A}\T\right) = \left( \partial\M{A}\right)\T$ &  \\
        $\PDer{\V{x}\T\V{y}}{\V{x}} = \PDer{\V{y}\T\V{x}}{\V{x}} = \V{y} $&\\
        $\PDer{\V{x}\T\M{A}\V{x}}{\V{x}} = \left(\M{A} + \M{A}\T\right) \V{x} $&    \\
   %     $\PDer{f\comp g}{x} =\left( \PDer{f}{x}\comp g \right)\, \PDer{g}{\V{x}}$ & chain rule\\
   %     $\PDer{f\comp g(\V{x})}{\V{x}} = \sum_m^M  \PDer{f}{\left[g(\V{x})\right]_m}\, \PDer{\left[g(\V{x})\right]_m}{\V{x}}$ & chain rule\\
        $\PDer{f\comp g(\V{x})}{\V{x}} = \At{\sum_{m=0}^{M-1}  \PDer{f}{u_m}\, \PDer{u_m}{\V{x}}}{\V{u} = g(\V{x})}$ & chain rule\\
        $\PDer{\left(\M{A}\,\V{x}-\V{y}\right)\T\M{B}\left(\M{A}\,\V{x}-\V{y}\right)}{\M{A}} = \left(\M{B} + \M{B}\T \right)\,\left(\M{A}\,\V{x}- \V{y}\right)\,\V{x}\T$&
        \end{tabular*}}
    \end{block}
     
    \begin{block}{Continuous Fourier transform}
    {\tablinesep=1.1ex
        \begin{tabular*}{\columnwidth}{@{\extracolsep{\fill}}l r }
    $\FT{f}(\nu) = \F{f}(\nu) = \int_{-\infty}^{+\infty} f(t) \E^{-\TwoIPi\,\nu\,t}\,\D{t}$& forward\\    
    $f(t) = \IF{\FT{f}}(t) = \int_{-\infty}^{+\infty} f(\nu) \E^{\TwoIPi\,\nu\,t}\,\D{\nu}$& inverse
        \end{tabular*}}
  %  {\bf Properties}
    {\tablinesep=1.1ex
        \begin{tabular*}{\columnwidth}{@{\extracolsep{\fill}}l r }
    $\F{\alpha f + \beta g} = \alpha \FT{f} + \beta \FT{g} $ & linearity\\    
    $\F{f(t-t_0)} =\E^{-\TwoIPi\,\nu\,t_0} \FT{f}(\nu) $ & shift\\    
    $\F{\E^{\TwoIPi\,\nu_0\,t} f(t) } = \FT{f}(\nu-\nu_0) $ & modulation\\   
    $\F{f(a\,t)} =\frac{1}{\Abs{a}} \FT{f}\left(\frac{\nu}{a}\right) $ & scaling\\
    $\F{f^*}(\nu) =\FT{f}^*(-\nu) $ & conjugation\\
    $ \int_{-\infty}^{+\infty} \Abs{\FT{f}(\nu)}^2\,\D{\nu} = \int_{-\infty}^{+\infty} \Abs{f(t)}^2\,\D{t} $ & Plancherel-Parseval\\
    $\FT{f}(0) = \int_{-\infty}^{+\infty} f(t)\,\D{t} $ & integration\\
    $\F{f^{(n)}} = \left(\TwoIPi\,\nu\right)^n\,\FT{f}(\nu)$ & differentiation\\
    $\F{f\ast g} =\FT{f}\,\FT{g}$ & convolution\\
    $\F{f\star f} = \Abs{\FT{f}}^2$ & autocorrelation
        \end{tabular*}}
    \end{block}
    
    \begin{block}{Discrete Fourier Transform}
    {\tablinesep=1.2ex
        \begin{tabular*}{\columnwidth}{@{\extracolsep{\fill}}l r }
            $ \M{F}=\begin{bmatrix}
                1&1&1&\cdots &1\\
                1&\omega &\omega^{2}&\cdots &\omega^{N-1}\\
                1&\omega^{2}&\omega^{4}&\cdots &\omega^{2(N-1)}\\
                % 1&\omega^{3}&\omega^{6}&\omega^{9}&\cdots &\omega^{3(N-1)}\\
                \vdots &\vdots &\vdots &\ddots &\vdots \\
                1&\omega^{N-1}&\omega^{2(N-1)}&\cdots &\omega^{(N-1)(N-1)}
                \end{bmatrix}$ & 
            \begin{tabular}{r} 
                with $\omega = \E^{-\frac{-2\imath\,\pi }{N}}$ \\
                the $N^\text{th}$ root of unity.
            \end{tabular}
        \end{tabular*}}    
        {\tablinesep=1.1ex
        \begin{tabular*}{\columnwidth}{@{\extracolsep{\fill}}l r }
         $\M{F}^{-1} = \frac{1}{N}\,\M{F}\T\,$ & orthogonality\\
         $\M{U} =  \frac{1}{\sqrt{N}}\,\M{F}$ & $\M{U}$ is  unitary\\
         $\Norm{\V{x}}^2_2 =\frac{1}{N}\Norm{\M{F}\,\V{x}}_2^2$& Plancherel-Parseval\\
         $\Norm{\V{x}}_1 \leq \Norm{\M{F}\,\V{x}}_1 \leq N\,\Norm{\V{x}}_1 $&
        \end{tabular*}}
    \end{block}
    
    \begin{block}{Circular convolution matrix $\M{H}$}
  {\tablinesep=1.1ex
        \begin{tabular*}{\columnwidth}{@{\extracolsep{\fill}}l r }
        $ H_{0,i} = h_i $ & impulse response (PSF)\\
        $\M{H}=\M{F}^{-1} \,\diag(\FT{\V{h}})\,\M{F}$& diagonalization by Fourier  \\
    $\FT{\V{h}}=\M{F}\,\V{h}\,\ $     & eigenvalues spectrum 
        \end{tabular*}}
    \end{block}
\end{column}

\begin{column}{0.5\textwidth - .5cm}

    \begin{block}{Continuous probability distribution}
    $\V{x}\in \Complexes^N$ is a continuous random vector, it has a probability density function (pdf) $f_X(\V{x})$ such that, for all $\Set{A} \subseteq  \Complexes^N$:
    \begin{align*}
        \Pr(\V{x}\in \Set{A})=\int_{\Set{A}} f_{X}(\V{x})\, \D{\V{x}}
    \end{align*}
  {\tablinesep=1.1ex
        \begin{tabular*}{\columnwidth}{@{\extracolsep{\fill}}l r }
       
        $\bar{\V{x}} = \Expect{\V{x}} = \Avg{\V{x}} = \int_{-\infty}^{+\infty} \V{x}\,f_X(\V{x})\,\D{\V{x}} $& Expectation  (mean)\\
        $\Expect{\alpha\,\V{x} +\beta\,\V{y} + \gamma} =\alpha\,\Expect{\V{x}} +\beta\,\Expect{\V{y}} + \gamma $ & linearity\\
        $\M{C}_{\V{x},\V{y}} = \Cov\left(\V{x},\V{y}\right) = \Expect{\V{x}\V{y}\T}  - \Expect{\V{x}}\,\Expect{\V{y}}\T$ & Cross-covariance \\
        $\M{C}_{\V{x}}=\Cov\left(\V{x},\V{x}\right) = \Expect{\V{x}\V{x}\T} -  \Expect{\V{x}}\,\Expect{\V{x}}\T$ & Covariance\\
        $\M{C}_{\V{x}}$ is an Hermitian matrix of size $N\times N$. 
        \end{tabular*}}
    \end{block}
    
    \begin{block}{Independence \& Uncorrelation}
      {\tablinesep=1.1ex
        \begin{tabular*}{\columnwidth}{@{\extracolsep{\fill}}l r }
        $ \Pr(X\Given Y) =  \Pr(X)$ & Independence \\
        $ f_{X,Y}(\V{x},\V{y}) = f_X(\V{x})\, f_Y(\V{y})$ & Independence\\
         $\Cov\left(\alpha\,\V{x} +\beta\,\V{y} + \gamma\right) = \alpha^2\,\Cov\left(\V{x}\right) +\beta^2\,\Cov\left(\V{y} \right)$ & Uncorrelation\\
        $ \Cov\left(\V{x},\V{y}\right) = 0 $ & Uncorrelation \\
        \end{tabular*}}
      {\tablinesep=1.1ex
        \begin{tabular*}{\columnwidth}{@{\extracolsep{\fill}}l r }
        independence $\Rightarrow$ uncorrelation & 
        uncorrelation $\nRightarrow$ independence  \\
        $  \Pr(X\Given Y) =  \frac{\Pr(Y\Given X)\,\Pr(X) }{\Pr(Y) }$& Bayes rule
        \end{tabular*}}\vspace{-1ex}
    \end{block}
    
  %  \begin{block}{Bayes rule}
 %   \begin{align*}
 %       \Pr(X\Given Y) =  \frac{\Pr(Y\Given X)\,\Pr(Y) }{\Pr(Y) }
 %   \end{align*}    
 %   \end{block}
    
    \begin{block}{Convexity}
    $ f: \Complexes^N \to \Reals $ is strictly convex if and only if:
    \begin{align*}
        f\left(\lambda\,\V{x} + (1-\lambda)\,\V{y}) < \lambda\,f(\V{x}) + (1-\lambda)\,f(\V{y}\right)\,\ \forall\,\V{x},\V{y}, \, \lambda\in ]0,1[\,.
    \end{align*}
%    $f$ has only a global minimum $\V{x}^+=\argmin_{\V{x}}  f(\V{x})$
%
      {\tablinesep=1.ex
        \begin{tabular*}{\columnwidth}{@{\extracolsep{\fill}}l r }
        $\V{x}^+=\argmin_{\V{x}}  f(\V{x})$& global minimum\\
 $ \V{p}\in\partial f(\V{x}) \Leftrightarrow f(\V{x})-f(\V{x}') \geq\Inner{\V{p}}{\V{x}-\V{x}'}\,, \forall\V{x}'$  & subgradient
        \end{tabular*}}\vspace{-1ex}
    \end{block}
    
    \begin{block}{Gradient descent}
    $f:  \Complexes^N \to \Reals $ is convex and differentiable with Lipschitz gradient $L$:
    \begin{align*}
        \Norm{\nabla \,\V{x} - \nabla \,\V{y}}_2 \leq L\,\Norm{\V{x} - \V{y}}_2
    \end{align*}
    The following sequence converges toward a minimizer of $f$ in $O(1/k)$: 
    \begin{align*}
        \V{x}^{(k+1)} = \V{x}^{(k)} - \gamma \nabla f(\V{x}^{(k)})  \text{ with } \gamma\in]0,1/L]
    \end{align*} 
    \end{block}
    
    \begin{block}{Newton's method}
    $f:  \Complexes^N \to \Reals $ is convex and twice  differentiable, the sequence:
    \begin{align*}
        \V{x}^{(k+1)} = \V{x}^{(k)} - \left(\nabla^2 f(\V{x}^{(k)})\right)^{-1}\,\nabla f(\V{x}^{(k)})  
    \end{align*} 
     converges toward a minimizer of $f$ in $O(1/k^2)$: 
    \end{block}
    
    
    \begin{block}{Projection}
    $\M{P}  \in \Complexes^{N\times N}$  is a projection on a subset $\Set{S} \subseteq   \Complexes^{N}$ and its indicator $\imath_{\Set{S}}$:
    
      {\tablinesep=1.1ex
        \begin{tabular*}{\columnwidth}{@{\extracolsep{\fill}}l r }
        $\M{P} \,\V{x} = \argmin_{\V{y}} \left(\imath_{\Set{S}}(\V{y}) + \frac{1}{2}\Norm{\V{x} - \V{y}}^2\right)$ & 
        with $\imath_{\Set{S}}(\V{x})\left\{
                            \begin{array}{ll}
                              0 &   \text{if } \V{x}\in\Set{S} \,, \\
                              +\infty & \text{otherwise} \,.
                            \end{array}
                                  \right.$\\
        $\M{P}\,\M{P} = \M{P}\,.$ & idempotent
        \end{tabular*}}
      {\tablinesep=1.1ex
        \begin{tabular*}{\columnwidth}{@{\extracolsep{\fill}}l r }
        $\Norm{\M{P} \,\V{x} - \M{P} \,\V{y}}_2 \leq \Norm{\V{x} - \V{y}}_2$ &  $\Set{S}$ is convex $\Rightarrow\M{P}$ is non-expansive\\
        $ \V{x}^{(k+1)} = \M{P}\left(\V{x}^{(k)} - \gamma \nabla f(\V{x}^{(k)}) \right)$ & projected gradient descent
        \end{tabular*}}
    \end{block}
    
    \begin{block}{Proximity operator}
    $f$: a lower semi-continuous convex function, its proximity operator is:
    \begin{align*}
        \prox_f(\V{y}) =   \argmin_{\V{y}} \left( f(\V{y}) + \frac{1}{2}\Norm{\V{x} - \V{y}}^2\right)
    \end{align*}
       $ \V{p} =  \prox_f(\V{x}) \Leftrightarrow \V{x} - \V{p} \in \partial f (\V{p})$  with $ \partial f (\V{p})$ the subgradient of $f$\\
        $\V{x}^+ = \prox_f(\V{x}^+)  \Leftrightarrow  \V{x}^+=\argmin_{\V{x}} f(\V{x})$% are the minimizers of $f$.
       % $\V{x}^{(k+1)} = \prox_f(\V{x}^{(k)})$  Algo du point proximal
    \end{block}\vspace{-1ex}
\end{column}
\end{columns}

\end{frame}
\end{document}
